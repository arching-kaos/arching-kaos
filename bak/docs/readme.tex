\documentclass[12pt]{report}
\usepackage{helvet}
\usepackage{biblatex}
\addbibresource{sample.bib}
\usepackage[utf8]{inputenc}
\usepackage[english,greek]{babel}
\usepackage[T1]{fontenc}
\title{Arching Kaos}
\author{Kaotisk Hund}
\begin{document}
\selectlanguage{english}
\maketitle
\begin{abstract}
Little words about what follows... arching ... kaos !!!
\end{abstract}
\tableofcontents
\selectlanguage{english}
\chapter{Introduction}
\chapter{Theory}
In following, we will overview some basic concepts such as radiowaves, what they are and how the are transmitted, radio stations through history, which was their role and which is this today, what do they need to operate and some theoritical organisation models of a hypothetical, commercial or not, radio station.

\section{Radiowaves}
Radio waves are radiated by electric charges undergoing acceleration\cite{first}.\footnote{Kraus, John D. (1988). Antennas, 2nd Ed. Tata-McGraw Hill. p. 50. ISBN 0074632191.; Serway, Raymond; Faughn, Jerry; Vuille, Chris (2008). College Physics, 8th Ed. Cengage Learning. p. 714. ISBN 978-0495386933.}. They are generated artificially by time varying electric currents, consisting of electrons flowing back and forth in a metal conductor called an antenna,[6][7] thus accelerating. In transmission, a transmitter generates an alternating current of radio frequency which is applied to an antenna. The antenna radiates the power in the current as radio waves. When the waves strike the antenna of a radio receiver, they push the electrons in the metal back and forth, inducing a tiny alternating current. The radio receiver connected to the receiving antenna detects this oscillating current and amplifies it.

As they travel farther from the transmitting antenna, radio waves spread out so their signal strength (intensity in watts per square meter) decreases, so radio transmissions can only be received within a limited range of the transmitter, the distance depending on the transmitter power, antenna radiation pattern, receiver sensitivity, noise level, and presence of obstructions between transmitter and receiver. An omnidirectional antenna transmits or receives radio waves in all directions, while a directional antenna or high gain antenna transmits radio waves in a beam in a particular direction, or receives waves from only one direction.

Radio waves travel through a vacuum at the speed of light, and in air at very close to the speed of light, so the wavelength of a radio wave, the distance in meters between adjacent crests of the wave, is inversely proportional to its frequency.

The other types of electromagnetic waves besides radio waves; infrared, visible light, ultraviolet, X-rays and gamma rays, are also able to carry information and be used for communication. The wide use of radio waves for telecommunication is mainly due to their desirable propagation properties stemming from their large wavelength.[7] Radio waves have the ability to pass through the atmosphere, foliage, and most building materials, and by diffraction can bend around obstructions, and unlike other electromagnetic waves they tend to be scattered rather than absorbed by objects larger than their wavelength.


\selectlanguage{greek}
Τα ραδιοκύματα ανακαλύφθηκαν από τον ... το ... με [...] (κάποιον τρόπο. Τα πρώτα πειράματα που έγιναν ήταν σχετικά με την [...]\footnote{πηγή εδώ}.
\selectlanguage{english}
\section{Radio Stations}
A radio broadcasting station is usually associated with wireless transmission, though in practice broadcasting transmission (sound and television) take place using both wires and radio waves. The point of this is that anyone with the appropriate receiving technology can receive the broadcast.[17]
Use of a sound broadcasting station

In line to ITU Radio Regulations (article1.61) each broadcasting station shall be classified by the service in which it operates permanently or temporarily.

\subsection{History}

\subsection{Operation}
\subsection{Work Models}
\subsection{Today}
\section{Music}

\chapter{Into the era of digital technology}
\section{Some stuff}
\section{Internet}
\section{Collaborative management}


\chapter{arching-kaos-docs}\label{arching-kaos-docs}

\begin{quote}
Documentation for installing and operating arching-kaos
\end{quote}

\section{Description}\label{description}

Arching Kaos is a decentralized radio station.

\section{Features}\label{features}

Among with a stream of mixes, there are some features that Arching Kaos
provides.

These are: 
\begin{itemize}
\item installing and operating a web radio station,
\item an API for uploading mixtape-like content,
\item listing of all radio mixtapes
\item registering uploaders,
\item a web site that renders based on live information through the API,
\item an IRC network, that also appears on the radio's website,
\item a decentralized sosial network for sharing the latest uploads and/or news and
\item decentralized filesystem storage for sharing and helping each other on file hosting.
\end{itemize}
\section{Repository usage}\label{repository-usage}

This is where the stream of changes are stored. You can clone them using
\texttt{git} to your local filesystem like this:

\begin{verbatim}
git clone <url>
\end{verbatim}

Edit install.sh to match your preferences. There
is a couple of variables being set there which are replaced by some
default values. Most of the values can work from the default with no
editing of the file, but it's good to have a look at it before running
it anyway.

Run \texttt{./install.sh} to configure, install and start
\texttt{arching-kaos}.

\section{How it works?}\label{how-it-works}

We can find out how it works but mostly we need to know what the
problems are that we are going to solve.

\subsection{Problems to solve}\label{problems-to-solve}


\subsubsection{1. 24/7 music}\label{music}

For that, we play all the mixtapes from a playlist in repeat, scanning
regularly for content.


\subsubsection{2. Multiple contributors}\label{multiple-contributors}

In a radio station and for a quality time and stream of music, it's
needed to have producers that they produce the content. So, we need to
be able to accept people's contributions to the radio station, in order
to play them and further share them.


\subsubsection{3. File storage}\label{file-storage}

As people host their files into their computers, so does a server. We
can utilize the existence of a file in two places or more places with
p2p technology. (See 4. File Sharing)


\subsubsection{4. File sharing}\label{file-sharing}

As a matter of fact, p2p technology is used for hosting information on
multiple peers which can be synchronized through the network. These
include IPFS, the InterPlanetery FileSystem which helps us on
transfering files big files with no limits, works as p2p so our
downloads can't fail as it rechecks each chunk of the file for its
consistency and provides an HTTP gateway for accessing files.


\subsubsection{5. Point of view}\label{point-of-view}

Another point is how the radio station and the so formentioned mixtapes
are going to make their way into publicity! So, apparently, since we
speak for a web radio, a web page would be perfect for that.


\subsubsection{6. Communications}\label{communications}

Music is about communication. A radio too. How people reach each other?
A network of users could be established for exchanging messages. So far,
IRC does a pretty good job on this. So we 'll be making use of an IRC
server in order to exchange instant messages.


\subsubsection{7. Socialization}\label{socialization}

Instant messaging is enough for getting information ``in the moment'',
but what about the updates of the station? There is a need for a place
that people can be informed about the latest mixtapes and the news of
the radio station.


\subsubsection{8. Pack all these}\label{pack-all-these}

Since, so far we made a list of 7 problems, of course a solution to all
these by one package is a matter of consideration. Organizing and
optimizing different components and ensure that will work similarly on
different environments is a priority. So we make an image of each
component and run containers with docker.


\subsection{Quick overview}\label{quick-overview}


\subsubsection{1. It's based on icecast2 server to produce a
stream.}\label{its-based-on-icecast2-server-to-produce-a-stream.}

We just initialize a very simple but configured from our
\texttt{./install.sh} script Icecast v2 docker image.


\subsubsection{2. Uses IPFS to upload, provide links and sync
mixes.}\label{uses-ipfs-to-upload-provide-links-and-sync-mixes.}

We use IPFS as a gateway to retrieve IPFS links to the local File System
of our working environment. We need to save in order to provide a ``safe
source'' for Liquidsoap (See 8. Liquidsoap). We also pin the links to
our IPFS repository.


\subsubsection{3. Uses DAT and Torrent to produce sync
options.}\label{uses-dat-and-torrent-to-produce-sync-options.}

Whenever a mixtape is synced to the radio station, a DAT repository (and
a torrent should but not yet) is created and stored in the mixtape list.


\subsubsection{4. Uses SSB to produce newsletter on SSB
network.}\label{uses-ssb-to-produce-newsletter-on-ssb-network.}

Whenever a mixtape is synced, SSB network is informed about it's artist,
title and IPFS link to listen, in \#arching-kaos-mixes.


\subsubsection{5. Uses ReactJS to produce a webpage for the radio
station.}\label{uses-reactjs-to-produce-a-webpage-for-the-radio-station.}

For creating a web interface, ReactJS framework was used, along with
some external libraries, like \texttt{node-fetch} to fetch statistics
(artist and title) for the currently playing mixtape on the streaming
radio station and for creating a list of all the mixtapes to listen on
demand.


\subsubsection{6. Uses expressJS to create and run an API that holds a list
of shows and an IP whitelist for authenticating
uploaders.}\label{uses-expressjs-to-create-and-run-an-api-that-holds-a-list-of-shows-and-an-ip-whitelist-for-authenticating-uploaders.}

ExpressJS is a nodeJS library that helps us create an HTTP server
instance which can be manually routed. What I do in there, is providing
routes that activate certain procedures. We use this for registering
uploaders, upload mixtapes and provide the mixtape list to whoever
applies for it.


\subsubsection{7. Uses CJDNS network for unique IP per
uploader.}\label{uses-cjdns-network-for-unique-ip-per-uploader.}

As most of the devices connected to the internet are mostly routed
through a NAT network, it's not possible to have different uploaders
registered through their IPs. CJDNS creates a p2p mesh network, which is
based on public/private key encryption. Every peer (user of this
program) has a unique cryptographic key pair which consists of a public
and a private key. The private one is used to sign outgoing packets and
decrypt incoming ones, while the public one is used to encrypt packets
for the particular peer. The public key is also transformed to a private
IPv6 on fc00::/7 block which can be then used for actually send and
receive packets from and to the network.


\subsubsection{8. Uses Liquidsoap to provide inputs to
icecast2.}\label{uses-liquidsoap-to-provide-inputs-to-icecast2.}

We already have a stream server, icecast v2, set up, waiting to server
streams to clients. But actually, there are no streams. What we are
going to about it, is use Liquidsoap for playing a playlist containing
the mixtapes, which is regularly refreshed and reload, to icecast's
specific mount point so listeners can reach to listen to.


\subsubsection{9. Uses charybdis-ircd for
communication.}\label{uses-charybdis-ircd-for-communication.}

Many different IRC daemons/servers were investigated to see what fits
best, in terms of connectivity, maintainance and the ability to make a
network of IRC servers. I choose charybdis-ircd for being the most
complete IRCv3. It can also connect to other charybdis-ircd instances
which is good option for creating a network of IRC servers to prevent
downtime and centrality problems.


\section{Environment preparation}\label{environment-preparation}

The final testing got part on a Gentoo Linux, a Linux distribution by
Gentoo Foundation Inc, but is easily adjusted to Fedora Linux which is
sponsored by Red Hat Inc.~Ubuntu Linux which is sponsored by Canonical
Inc, is compatible as well, but most of the automatic tools are bundled
with Fedora's package manager, \texttt{dnf}.


\subsection{Off to go!}\label{off-to-go}

We are on a freshly installed Fedora Linux, that's not a requirement, it
can be any other distribution but I 'll use this as a reference.

We need to install basic packages to get the tools will use later. These
three things, \texttt{cjdns}, \texttt{docker-ce},
\texttt{docker-compose}, \texttt{git}, \texttt{bison},
\texttt{automake}, \texttt{libtool}, \texttt{m4} and \texttt{nginx}.

\#TODO Make sure there is a script for all these, at least on Fedora,
plus, remember what \texttt{bison} was needed for\ldots{} charybdis?

After installing these, we need to clone this repository, if not already
cloned, and change directory to it

\begin{verbatim}
git clone https://git.arching-kaos.net/kaotisk/arching-kaos.git
cd arching{-}kaos
\end{verbatim}

The installation makes a directory \texttt{./storage} with all the data.

Note: both the projects' files and data files are used. The
\texttt{./storage} folder is excluded from the repository but not
\texttt{./etc}. If you are contributing be sure not to publish
passwords.


\section{How the installation
works}\label{how-the-installation-works}

After cloning the repository, you can review the \texttt{./install.sh}
file, in order to configure basic parts of arching-kaos.

We will now go into a deep analysis of the \texttt{./install.sh} script
and see what it does and also explain the values that can be edited in
there.

We start with a small greeting to the user:

\begin{verbatim}
#!/bin/sh
echo "Getting the basics done..."
echo "Initializing and updating modules..."
\end{verbatim}


\subsection{Submodules initiation}\label{submodules-initiation}

In the very first part of \texttt{./install.sh} we just pull the git
modules that we want to include in our project. To achieve that, we
change our working directory to \texttt{./modules} where the modules are
supposed to get downloaded.

\begin{verbatim}
cd modules
\end{verbatim}

We are going to start with \texttt{arching-kaos-api}.

\begin{verbatim}
git submodule init arching-kaos-api
git submodule update arching-kaos-api
\end{verbatim}

We continue with \texttt{arching-kaos-radio}.

\begin{verbatim}
git submodule init arching-kaos-radio
git submodule update arching-kaos-radio
\end{verbatim}

Add the \texttt{arching-kaos-generic}.

\begin{verbatim}
git submodule init arching-kaos-generic
git submodule update arching-kaos-generic
\end{verbatim}

Also the \texttt{arching-kaos-ssb}.

\begin{verbatim}
git submodule init arching-kaos-ssb
git submodule update arching-kaos-ssb
\end{verbatim}

The \texttt{arching-kaos-irc}.

\begin{verbatim}
git submodule init arching-kaos-irc
git submodule update arching-kaos-irc
\end{verbatim}

And the \texttt{docker-dat-store}

\begin{verbatim}
git submodule init docker-dat-store
git submodule update docker-dat-store
\end{verbatim}

We inform that we are done with the above

\begin{verbatim}
echo "Done!"
\end{verbatim}

And we change back to our root directory (project's one).

\begin{verbatim}
cd ..
\end{verbatim}


\subsection{Configuring ./etc}\label{configuring-.etc}

Now, what we are going to do is simply replace variables placed in
various files across the project to achieve correct settings before we
run anything.

\begin{verbatim}
echo "Configuring ./etc ..."
\end{verbatim}


\subsubsection{IRC}\label{irc}

We start with charybdis IRC daemon settings. Please adjust to your
needs.

\begin{verbatim}
echo "...1/4 charybdis"
\end{verbatim}

There are certain settings that need to be configured in charybdis. Our
main approach is to have everything ready to get into a working state.
In \texttt{./install.sh} the following blocks from
\texttt{./etc/charybdis/ircd.conf} are going to be edited. Comments and
defaults are stripped. See actual file for more information and help.

\begin{itemize}

\item
  Server information
\end{itemize}

\begin{verbatim}
serverinfo {
  name = "{$IRC_NAME}";
  sid = "{$IRC_SID}";
  description = "{$IRC_DESCRIPTION}";
  network_name = "{$IRC_NETNAME}";
  vhost = "{$PUBLIC_IPV4}";
  # In case you want to use a public IPv6 you may want
  # to uncomment the following line and also set an IPv6
  # in the appropriate sed command.
  #vhost6 = "{$PUBLIC_IPV6}";
  vhost6 = "{$CJDNS_IPV6}";
};
\end{verbatim}

\begin{itemize}

\item
  Administrator information
\end{itemize}

\begin{verbatim}
admin {
  name = "{$ADMIN_NAME}";
  description = "{$ADMIN_DESCRIPTION}";
  email = "{$ADMIN_EMAIL}";
};
\end{verbatim}

\begin{itemize}

\item
  Listening settings
\end{itemize}

\begin{verbatim}
listen {
  host = "{$PUBLIC_IPV4}";
  sslport = 6697;
  host = "{$CJDNS_IPV6}";
  port = 6667;
};
\end{verbatim}

\begin{itemize}

\item
  Operator settings
\end{itemize}

\begin{verbatim}
auth {
  password = "{$IRC_AUTH_PASSWORD}";
  class = "opers";
};
\end{verbatim}

\begin{itemize}

\item
  Administrator settings
\end{itemize}

\begin{verbatim}
operator "god" {
  user = "*god@{$CJDNS_IPV6}";
  password = "{$GOD_IRC_PASSWORD}";
  snomask = "+Zbfkrsuy";
  flags = ~encrypted;
  privset = "admin";
};
\end{verbatim}

We use \texttt{sed} to set the variables (\{\$variables\}) to values.

\begin{verbatim}
sed -i.bak -e 's/{$IRC_NAME}/irc.arching-kaos.net/' \
  etc/charybdis/ircd.conf
sed -i.bak -e 's/{$IRC_SID}/44Q/' etc/charybdis/ircd.conf
sed -i.bak -e 's/{$IRC_DESCRIPTION}/A friendly IRC server/' \
  etc/charybdis/ircd.conf
sed -i.bak -e 's/{$IRC_NETNAME}/irc.arching-kaos.net/' \
  etc/charybdis/ircd.conf
sed -i.bak -e 's/{$PUBLIC_IPV4}/127.0.0.1/g' \
  etc/charybdis/ircd.conf
# In case you have a public IPv6 you should just insert 
# it between the double slashes.
# sed -i.bak -e 's/{$PUBLIC_IPV6}//g' etc/charybdis/ircd.conf
sed -i.bak \
  -e \
  's/{$CJDNS_IPV6}/fc42:7cfa:b830:e988:f192:717f:6576:ed12/g' \
  etc/charybdis/ircd.conf
sed -i.bak -e 's/{$ADMIN NAME}/kaotisk/' \
  etc/charybdis/ircd.conf
sed -i.bak -e 's/{$ADMIN_DESCRIPTION}/some descr/' \
  etc/charybdis/ircd.conf
sed -i.bak -e 's/{$ADMIN_EMAIL}/kaotisk@arching-kaos.com/' \
  etc/charybdis/ircd.conf
sed -i.bak -e 's/{$IRC_AUTH_PASSWORD}/somepass/' \
  etc/charybdis/ircd.conf
sed -i.bak -e 's/{$GOD_IRC_PASSWORD}/somepass/' \
  etc/charybdis/ircd.conf
\end{verbatim}


\subsubsection{Icecast2}\label{icecast2}

And we continue with icecast2 in which we set in the same way the
following variables in \texttt{./etc/icecast2/icecast.xml} file. As the
suffix of the filename implies, its an XML\footnote{https://www.w3.org/XML} file, hence it's
consisting of tags with values (e.g. <tag>value</tag>).

\begin{verbatim}
echo "...2/4 icecast"
\end{verbatim}

\begin{itemize}

\item
  Location We can set our instance's location.
\end{itemize}

\begin{verbatim}
<location>{$LOCATION}</location>
\end{verbatim}

\begin{itemize}

\item
  Administrator email We can set an administrator email. It shows up on
  the icecast web page under the Server Information section of page
  Version.
\end{itemize}

\begin{verbatim}
<admin>{$ADMIN_EMAIL}</admin>
\end{verbatim}

\begin{itemize}

\item
  Authentication
\end{itemize}

\begin{verbatim}
<authentication>
  <!-- Sources log in with username 'source' -->
  <source-password>{$ICECAST_SOURCE_PASSWORD}</source-password>
  <!-- Relays log in with username 'relay' -->
  <relay-password>{$ICECAST_RELAY_PASSWORD}</relay-password>
  <!-- Administrators log in with username 'admin' alias -->
  <admin-user>admin</admin-user>
  <admin-password>{$ICECAST_ADMIN_PASSWORD}</admin-password>
</authentication>
\end{verbatim}

\begin{itemize}

\item
  Hostname
\end{itemize}

\begin{verbatim}
<hostname>{$ICECAST_HOSTNAME}</hostname>`
\end{verbatim}

\begin{itemize}

\item
  Setting our webradio domain name to be able to access icecast's stats
\end{itemize}

\begin{verbatim}
<http-headers>
<header name="Access-Control-Allow-Origin" \
  value="{$RADIO_WEBSITE_BASEURL}" />
</http-headers>
\end{verbatim}

So we do \texttt{sed} again. These are the variables that you can edit accordingly in \texttt{./install.sh}.

\begin{verbatim}
sed -i.bak -e 's/{$LOCATION}/earth/' etc/icecast2/icecast.xml
sed -i.bak -e 's/{$ADMIN_EMAIL}/kaotisk@arching-kaos.com/' \
  etc/icecast2/icecast.xml
sed -i.bak -e 's/{$ICECAST_SOURCE_PASSWORD}/hackme/' \
  etc/icecast2/icecast.xml
sed -i.bak -e 's/{$ICECAST_RELAY_PASSWORD}/hackme/' \
  etc/icecast2/icecast.xml
sed -i.bak -e 's/{$ICECAST_ADMIN_PASSWORD}/hackme/' \
  etc/icecast2/icecast.xml
sed -i.bak -e \
's/{$ICECAST_HOSTNAME}/icecast.arching-kaos.local/' \
  etc/icecast2/icecast.xml
sed -i.bak -e \
  's/ \
  {$RADIO_WEBSITE_BASEURL}/ \
  http:\/\/radio.arching-kaos.local/' \
  etc/icecast2/icecast.xml
\end{verbatim}


\subsubsection{Liquidsoap}\label{liquidsoap}

There goes liquidsoap settings also. We basically need to edit two lines:

\begin{verbatim}
echo "...3/4 liquidsoap"
\end{verbatim}

\begin{itemize}

\item
  Source icecast password
\end{itemize}

\begin{verbatim}
output.icecast(
  %vorbis,
  radio,
  mount="demo.ogg",
  host="localhost",
  password="{$ICECAST_SOURCE_PASSWORD}",
  url="http://radio.arching-kaos.com",
  description="Arching Kaos radio is a collaborative webradio."
)
\end{verbatim}

\begin{itemize}

\item
  Live stream access
\end{itemize}

\begin{verbatim}
live = input.harbor(
  "live",
  port=3210,
  user="source",
  password="{$LIVE_SOURCE_PASSWORD}"
) 
\end{verbatim}

Which we also edit with \texttt{sed}.

\begin{verbatim}
sed -i.bak -e 's/{$ICECAST_SOURCE_PASSWORD}/hackme/' \
  etc/liquidsoap/radio.liq
sed -i.bak -e 's/{$LIVE_SOURCE_PASSWORD}/hackmetoo/' \
  etc/liquidsoap/radio.liq
\end{verbatim}


\subsubsection{NGINX and intergration}\label{nginx-and-intergration}

\begin{verbatim}
echo "...4/4 nginx"
\end{verbatim}

We use nginx to serve our pages through the network.
\texttt{./etc/conf.d} files as well source files from the submodules
\texttt{./modules} that use the same variables are set in the following
way. - API

\begin{verbatim}
sed -i.bak -e 's/{$API_SERVER_NAME}/api.arching-kaos.local/g' \
  etc/nginx/conf.d/api.conf \
  modules/arching-kaos-radio/src/ShowList.js \
  modules/arching-kaos-radio/src/Menu.js
\end{verbatim}

\begin{itemize}

\item
  Documentation
\end{itemize}

\begin{verbatim}
sed -i.bak \
  -e 's/{$DOCS_SERVER_NAME}/docs.arching-kaos.local/g' \
  etc/nginx/conf.d/api.conf \
  etc/nginx/conf.d/docs.conf
\end{verbatim}

\begin{itemize}

\item
  Generic domain name
\end{itemize}

\begin{verbatim}
sed -i.bak -e 's/{$DOMAIN_NAME}/arching-kaos.local/g' \
  etc/nginx/conf.d/default.conf \
  modules/arching-kaos-radio/src/Signature.js
\end{verbatim}

\begin{itemize}

\item
  Icecast
\end{itemize}

\begin{verbatim}
sed -i.bak \
  -e 's/{$ICECAST_SERVER_NAME}/icecast.arching-kaos.local/g' \
  etc/nginx/conf.d/icecast.conf \
  modules/arching-kaos-radio/src/App.js \
  modules/arching-kaos-radio/src/Menu.js \
  modules/arching-kaos-radio/src/NowPlaying.js
\end{verbatim}

\begin{itemize}

\item
  IPFS
\end{itemize}

\begin{verbatim}
sed -i.bak \
  -e 's/{$IPFS_SERVER_NAME}/ipfs.arching-kaos.local/g' \
  etc/nginx/conf.d/ipfs-gateway.conf \
  modules/arching-kaos-api/config.js
\end{verbatim}

\begin{itemize}

\item
  IRC

  \begin{itemize}

  \item
  Server settings
  \end{itemize}

\begin{verbatim}
sed -i.bak \
  -e 's/{$IRC_SERVER_NAME}/irc.arching-kaos.local/g' \
  etc/nginx/conf.d/irc.conf \
  etc/thelounge/config.js
\end{verbatim}

  \begin{itemize}

  \item
  Client settings
  \end{itemize}

\begin{verbatim}
sed -i.bak \
  -e 's/{$IRC_CLIENT}/http:\/\/127.0.0.1:9000/g' \
  modules/arching-kaos-radio/src/Chat.js \
  modules/arching-kaos-irc/index.html
\end{verbatim}
\item
  Radio
\end{itemize}

\begin{verbatim}
sed -i.bak \
  -e 's/{$RADIO_SERVER_NAME}/radio.arching-kaos.local/g' \
  etc/nginx/conf.d/radio-arching.conf \
  modules/arching-kaos-radio/src/Header.js
\end{verbatim}

\begin{itemize}

\item
  SSB
\end{itemize}

\begin{verbatim}
sed -i.bak -e 's/{$SSB_SERVER_NAME}/ssb.arching-kaos.local/g' \
  etc/nginx/conf.d/ssb.conf \
  etc/ssb-pub-data/config
\end{verbatim}

\begin{itemize}

\item
  Opentracker
\end{itemize}

\begin{verbatim}
sed -i.bak -e \
   's/{$TRACKER_SERVER_NAME}/tracker.arching-kaos.local/' \
   etc/nginx/conf.d/tracker.conf
\end{verbatim}


\subsubsection{API}\label{api}

We, then, create a folder where the \texttt{arching-kaos-api} files will
reside when we will run it. And also, copy some sample files for getting
the API ready.

\begin{verbatim}
echo "Create API directories"
# sh ./modules/arching-kaos-api/api-dir.sh 
# Going the custom way again
export ARCHING_KAOS_API_DIR=$PWD/storage/.arching-kaos-api
mkdir -p $ARCHING_KAOS_API_DIR/downloads
cp modules/arching-kaos-api/ipList.json-sample \
  $ARCHING_KAOS_API_DIR/ipList.json
cp modules/arching-kaos-api/shows.json-sample \
  $ARCHING_KAOS_API_DIR/shows.json
\end{verbatim}


\subsection{Starting docker images or build them in some
cases}\label{starting-docker-images-or-build-them-in-some-cases}

In folder \texttt{./scripts/} files starting with \texttt{docker-} are
small scripts that start certain images for our project to work.

\begin{verbatim}
echo "Getting docker scripts ready ..."
echo "Proceeding arching-kaos installation ..."
\end{verbatim}


\subsubsection{Documentation}\label{documentation}

So we basically want to have our source documentation available so we
start it from a script.

\begin{verbatim}
echo "Starting docs..."
sh ./scripts/docker-arching-kaos-docs.sh
echo "... done"
\end{verbatim}


\subsubsection{Icecast2}\label{icecast2-1}

\begin{verbatim}
echo "Starting icecast..."
sh ./scripts/docker-icecast.sh
echo "... done"
\end{verbatim}


\subsubsection{IPFS}\label{ipfs}

\begin{verbatim}
echo "Starting ipfs..."
sh ./scripts/docker-ipfs.sh
echo "... done"
\end{verbatim}


\subsubsection{Opentracker}\label{opentracker}

\begin{verbatim}
echo "Starting opentracker..."
sh ./scripts/docker-opentracker.sh
echo "... done"
\end{verbatim}


\subsubsection{SSB}\label{ssb}

\begin{verbatim}
echo "Starting ssb..."
sh ./scripts/docker-ssb-create.sh
echo "... done"
\end{verbatim}


\subsubsection{Liquidsoap}\label{liquidsoap-1}

\begin{verbatim}
echo "Starting liquidsoap..."
sh ./scripts/docker-liquidsoap.sh
echo "... done"
\end{verbatim}


\subsubsection{API}\label{api-1}

\begin{verbatim}
echo "Starting API..."
cd modules/arching-kaos-api
sh ./install.sh
cd ../..
sh ./modules/arching-kaos-api/run.sh
echo "... done"
\end{verbatim}


\subsubsection{Web interface for the
radio}\label{web-interface-for-the-radio}

\begin{verbatim}
echo "Starting webpage..."
cd modules/arching-kaos-radio
./start.sh
echo "... done"
cd ../..
\end{verbatim}


\subsubsection{dat-store}\label{dat-store}

\begin{verbatim}
echo "Starting dat-store..."
cd modules/docker-dat-store
sh ./build.sh
sh ./start.sh
echo "... done"
cd ../..
\end{verbatim}


\subsubsection{thelounge}\label{thelounge}

\begin{verbatim}
echo "Starting thelounge..."
sh ./scripts/docker-thelounge.sh
echo "... done"
\end{verbatim}


\subsection{Building last
components}\label{building-last-components}


\subsubsection{IRC (charybdis)}\label{irc-charybdis}

\begin{verbatim}
echo "Setting up IRC"
sh ./scripts/charybdis-simple-install.sh
cp etc/charybdis/ircd.conf $HOME/ircd/etc/ircd.conf
cp etc/charybdis/ircd.motd $HOME/ircd/etc/ircd.motd
echo "Starting IRC..."
$HOME/ircd/bin/charybdis
#TODO Insert crontab @reboot
\end{verbatim}


\subsection{Patching everything into NGINX web
proxy}\label{patching-everything-into-nginx-web-proxy}

\begin{verbatim}
echo "Starting NGINX..."
docker run --name nginx \
  --restart always \
  -d --network=host \
  -v $PWD/etc/nginx/conf.d:/etc/nginx/conf.d \
  -v $PWD/modules/arching-kaos-generic:/srv/generic \
  -v $PWD/modules/arching-kaos-irc:/srv/irc \
  -v $PWD/modules/arching-kaos-ssb:/srv/ssb \
  nginx
\end{verbatim}


\subsection{Edit SSB landpage}\label{edit-ssb-landpage}

Execute the following to get your SSB's servers public id:
\begin{verbatim}
docker exec -it sbot sbot whoami > my_ssb_ident
\end{verbatim}

and save it to file \texttt{my\_ssb\_ident}. Also execute to get a code
for 100 invites it:
\begin{verbatim}
docker exec -it sbot \
  sbot invite.create 100 > invite_code
\end{verbatim}
apply the info above in \texttt{./modules/arching-kaos-ssb/index.html}.


\section{There you go!}\label{there-you-go}

Our application is successfully installed and running!!

\begin{verbatim}
echo "Voila!"
\end{verbatim}
\medskip
\printbibliography[heading=bibintoc,title={Whole bibliography}]

\end{document}
